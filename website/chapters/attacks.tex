\section{Attacks}
\label{sec:attacks}
% Against which attacks should these security services protect the system?

\begin{itemize}
    \item \textbf{Passive attacks}
    \begin{enumerate}
        \item Eavesdropping
        \begin{enumerate}
            \item A third party wants to leak the current intermediate election results to influence the further voting process.
        \end{enumerate}
        
        \item Traffic analysis
        \begin{enumerate}
            \item \textit{Not applicable.}
        \end{enumerate}
    \end{enumerate}
    
    \item \textbf{Active attacks}
    \begin{enumerate}
        \item Message insertion/modification:
        \begin{enumerate}
            \item A third party want to change votes to a certain political party.
        \end{enumerate}
        
        \item Impersonation/masquerade:
        \begin{enumerate}
            \item A third party want to repeat votes on a certain political party.
            \item A third party want to reduce the votes for certain political parties.
        \end{enumerate}
        
        \item Replay:
        \begin{enumerate}
            \item A third party want to repeat votes on a certain political party.
        \end{enumerate}
        
        \item (Distributed) Denial-of-Service ((D)Dos):
        \begin{enumerate}
            \item A third party wants to boycott the elections.
        \end{enumerate}
        
        \item Hijacking (taking over existing connection):
        \begin{enumerate}
            \item Man-in-the-middle (MITM) attacks: a third party pretends to be the voting website to vote with the visitors credentials.
            \item Man-in-the-browser (MITB) attacks: malware that attacks the website in the user's browser.
        \end{enumerate}
    \end{enumerate}
\end{itemize}