\documentclass[a4paper,12pt]{article}

% Packages
\usepackage[english]{babel}
\usepackage[utf8x]{inputenc}
\usepackage[T1]{fontenc}
\usepackage{helvet}
\renewcommand{\familydefault}{\sfdefault}
\usepackage[margin=1in]{geometry}
\usepackage{parskip}
\usepackage{hyperref}
\usepackage{enumerate}
\usepackage{graphicx}
\usepackage{amsmath}
\usepackage{amssymb}
\graphicspath{ {./images} }
\usepackage[numbers]{natbib}
\bibliographystyle{IEEEtranN}

\title{Securing Web-based Electronic Elections}
\author{Bram De Smet \\ \small Bram.DeSmet@UGent.be \and Robbe De Vilder \\ \small Robbe.DeVilder@UGent.be \and Jaime Roelandts \\ \small Jaime.Roelandts@UGent.be \and Garben Tanghe \\ \small Garben.Tanghe@UGent.be \and Thomas Uyttenhove \\ \small Thomas.Uyttenhove@UGent.be \and Olivier Van den Nest \\ \small Olivier.VandenNest@UGent.be}
\date{\today}

% Documents
\begin{document}

% Title
\maketitle

\newpage

% Abstract
\begin{abstract}
\noindent
One might have noticed that Belgian politicians are currently in ``election mode''. End of May 2019, the masses will set direction towards an Electoral Office to cast their vote. People need to volunteer to supervise the correct voting procedures, and all votes are manually counted. All of these procedures currently deployed might seem like a task screaming for automation and benefiting from a web-based approach.

\medskip
\noindent
However, it is key to thoroughly analyze the viability of such a system first. Your task is to investigate what it would take, if at all possible, to implement a web-based electronic voting system in a secure, fair and democratic manner. The goal of this assignment is to focus on the security aspects of online elections, and the arguments against or in favour of such an approach in comparison to the traditional voting process.
\end{abstract}

% Table of contents
\tableofcontents

\newpage
\section{Introduction}
\label{sec:introduction}
% 
Login using eID, assuming that the country has an eID system.

A Belgian citizen, should use his identity card to log in on to the website.
Once logged-in, he could make his vote and save it.
The website will keep a list of all votes, while keeping it anonymous and it will also keep a list of the people that did vote.

The following assumptions are made:
\begin{itemize}
    \item A voter can only cast their vote once.
    \item Every voter has access to an eID reader or is welcome to use a voting system provided by the government.
    \item No proxy voting can be performed.
    \item The network will be available during the voting period without interruption.
    \item The power supply won't be interrupted.
\end{itemize}

The proof-of-concept demonstration website is created using Node.js and the Express package.
For the database MongoDB is used with the \textbf{mongodb} package (\url{https://mongodb.github.io/node-mongodb-native/}).
To ensure security and encryption the following packages are used:
\begin{itemize}
    \item \textbf{crypto} (built-in) (\url{https://nodejs.org/api/crypto.html})
    \item \textbf{https} (\url{https://nodejs.org/api/https.html})
    \item \textbf{node-rsa} (\url{https://www.npmjs.com/package/node-rsa})
\end{itemize}
\section{Services}
\label{sec:services}
% Which services are required (confidentiality, authentication, data-integrity, non-repudiation, etc.)
\begin{enumerate}
    % Data can only be read by those who are allowed to read these data
    \item \textbf{Confidentiality}
    \begin{enumerate}
        \item The vote should be private.
        \item Backtracking to the author should not be possible.
    \end{enumerate}
    
    % Guaranteeing the authenticity of a communication
    \item \textbf{Authentication}
    \begin{enumerate}
        \item Only one vote per person can be cast.
        \item Someone should not be be able to vote on one's behalf.
    \end{enumerate}
    
    % Determines which user may access which resources
    \item \textbf{Access control / authorization}
    \begin{enumerate}
        \item The user must be entitled to vote.
        \item Voting should happen within allowed time span.
    \end{enumerate}
    
    % Guarantee that sent data and received data are identical
    \item \textbf{Data integrity}
    \begin{enumerate}
        \item The table is sticky, a cast vote cannot be changed.
        \item One's vote should not be counted twice.
    \end{enumerate}
    
    % For sender:
    % - Sender can’t deny having sent the message
    % - Important for receiver
    %
    % For receiver
    % - Receiver can’t deny having received the message
    % - Important for sender
    \item \textbf{Non-repudiation}
    \begin{enumerate}
        \item Whether or not a person has voted, not which party was voted for.
        \item From government's perspective a vote can not be considered invalid if it has been correctly cast.
        \item The user should have a certificate to prove he has voted (in case the server database crashes)...
        \begin{enumerate}
            \item The server makes a certificate with the users credentials.
            \item The server signs this certificate using his private key.
            \item The server encrypts the certificate using the users public key.
            \item The user receives the certificate for later proof.
        \end{enumerate}
    \end{enumerate}
    
    % System / service is accessible and usable for authorized users
    \item \textbf{Availability}
    \begin{enumerate}
        \item Must be accessible during the complete voting day, since cast votes can't be lost.
        \item The websites/databases will be decentralized. Multiple servers will set-up, in case a server is down the other can be accessed and the voting can continue. 
        \item Some public spaces should be provided with computers for people that do not have direct access to an internet connection.
    \end{enumerate}
\end{enumerate}
\section{Attacks}
\label{sec:attacks}
% Against which attacks should these security services protect the system?

\begin{itemize}
    \item \textbf{Passive attacks}
    \begin{enumerate}
        \item Eavesdropping
        \begin{enumerate}
            \item A third party wants to leak the current intermediate election results to influence the further voting process.
        \end{enumerate}
        
        \item Traffic analysis
        \begin{enumerate}
            \item \textit{Not applicable.}
        \end{enumerate}
    \end{enumerate}
    
    \item \textbf{Active attacks}
    \begin{enumerate}
        \item Message insertion/modification:
        \begin{enumerate}
            \item A third party want to change votes to a certain political party.
        \end{enumerate}
        
        \item Impersonation/masquerade:
        \begin{enumerate}
            \item A third party want to repeat votes on a certain political party.
            \item A third party want to reduce the votes for certain political parties.
        \end{enumerate}
        
        \item Replay:
        \begin{enumerate}
            \item A third party want to repeat votes on a certain political party.
        \end{enumerate}
        
        \item (Distributed) Denial-of-Service ((D)Dos):
        \begin{enumerate}
            \item A third party wants to boycott the elections.
        \end{enumerate}
        
        \item Hijacking (taking over existing connection):
        \begin{enumerate}
            \item Man-in-the-middle (MITM) attacks: a third party pretends to be the voting website to vote with the visitors credentials.
            \item Man-in-the-browser (MITB) attacks: malware that attacks the website in the user's browser.
        \end{enumerate}
    \end{enumerate}
\end{itemize}
\section{Countermeasures}
\label{sec:countermeasures}
% Which countermeasures have been taken against these attacks?
\begin{itemize}
    \item \textbf{Passive attacks}
    \begin{enumerate}
        \item Eavesdropping
        \begin{enumerate}
            \item Encryption of authentication data and the user's vote.
        \end{enumerate}
        
        \item Traffic analysis
        \begin{enumerate}
            \item \textit{Not applicable.}
        \end{enumerate}
    \end{enumerate}
    
    \item \textbf{Active attacks}
    \begin{enumerate}
        \item Message insertion/modification:
        \begin{enumerate}
            \item Digital signature of the user.
        \end{enumerate}
        
        \item Impersonation/masquerade:
        \begin{enumerate}
            \item Digital signature with private key from eID.
        \end{enumerate}
        
        \item Replay:
        \begin{enumerate}
            \item Server-side package dropping based on presence of entry in database.
        \end{enumerate}
        
        \item (Distributed) Denial-of-Service ((D)Dos):
        \begin{enumerate}
            \item Use third party service that prevents this issue to redirect traffic to the voting application.
        \end{enumerate}
        
        \item Hijacking (taking over existing connection):
        \begin{enumerate}
            \item SSL/TLS certificates used in the HTTPS protocol.
        \end{enumerate}
    \end{enumerate}
\end{itemize}
\section{Limitations and Vulnerabilities}
\label{sec:limitations_vulnerabilities}
% What are possible limitations and remaining vulnerabilities of your system?

\begin{itemize}
    \item Tables in database with votes can easily be altered by trusted people with authorized access.
    This is no different from the current voting system and this problem has not been solved by this implementation.
    \item Someone can still steal a user's eID and threaten them in order to get their PIN code.
    \item The system, as any web application, can only work if a stable power source and internet connection can be provided to the user. This responsibility does not lie with us, but with the power and telecom providers in Belgium.
\end{itemize}


\section{Security Mechanisms}
\label{sec:security_mechanisms}
% Which concrete security mechanisms (encryption algorithms, key lengths, etc.) do you use to implement these security services? Be sufficiently specific in the description of your choice.
Cloudflare (D)DoS protection.

HTTPS connection using TLS 1.2 or 1.3.
\section{Conclusion}
\label{sec:conclusion}
% Do not forget to write a conclusion to the assignment report.



% Bibliography
\phantomsection
\nocite{*}
\addcontentsline{toc}{section}{References}
\bibliography{references}
% Do not forget to mention the sources of your inspiration in the references.

\end{document}